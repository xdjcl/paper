\documentclass{XDUthesis} %[draft
\usepackage{algorithmic,algorithm}
\usepackage{graphicx}
\usepackage{caption}
\usepackage{subcaption}
\usepackage{hyperref}
\usepackage{float}
\usepackage{subcaption}
\usepackage{bm}

\newtheorem{mydef}{\emph{Definition}}
\restylefloat{table}

\title{科学数据的序列分类}
\author{刘精昌}
\date{}
%题目拆分:用于封面
\septitleA{科学数据的序列分类}%如果论文题目长度<=11中文字符只需填此项即可B项空着
\septitleB{}%如果论文题目长度>11中文字符, 建议拆分为10+x or 11+x 将剩余x个字符填在此处。
\class{071271}%班级号
\schoolnumber{07127026}%学号
\major{统计学}%专业
\school{数学与统计学院}%学院
\supervisor{徐林莉}%指导老师

\begin{document}

% 摘要
% 中文摘要
\begin{abstract}

    序列数据在日常生活中特别常见,比如心电图、语音序列等。对序列数据的分类对科学研究有很大的帮助,比如对心电图的分类能得知病人是否患病,语音序列的分类能对语音的来源进行识别。

    不同于一般的数据,序列数据一般有很多元素,而且各个元素之间都有一定的依赖关系,因而使用常见的分类方法,比如支持向量机,并不可行。常用的方法是定义序列之间的距离,根据序列之间的远近,把序列归类到离其最近的那个序列。一般序列之间的距离定义为欧式距离,欧式距离在求解时,只会把相同索引的元素对齐。但是比较序列相近与否时,我们更为看重的是序列形态的相似,而相近形态的序列片段大多不在序列的相同位置。DTW方法就能够有效解决欧式距离的这一缺点。可以定义反映两序列元素距离以及对齐关系的距离矩阵,DTW方法的关键就是找到从距离矩阵左下角游走到右上角的路径,两序列之间的DTW距离通过简单的迭代式就可以得到。此外,本文还具体介绍了一些DTW方法的加速技巧。通过人造数据以及官方数据的实验,可以发现,在大多数的情况下,DTW距离的分类精确度都要优于欧式距离。

    DTW方法考虑的是序列整体,本文还介绍了一种利用序列局部特性进行序列分类的方法——Shapelets。该方法需要找到最能体现序列特性的子序列Shapelets,再根据待分类序列最具特征的局部片段和Shapelets的距离,对待分类序列进行分类。

\end{abstract}
\keywords{序列分类, 序列距离, DTW方法, Shapelets}

% 英文摘要
\begin{enabstract}
  Sequence data is so common, like ECG, speech sequence. Classifying sequence will be helpful for scientific research, we can know suspicious patient sick or not from ECG and recognize speech by classifying speech sequence.

  Different from general data, there are so many elements in sequences and these elements are not independent, so some usual methods, such as SVM, will not workable. Generally, we need to define distance between two sequences, and classify the sequence to the nearest one. Commonly, the distance is defined as Euclidean distance, which can only align elements with the same index. But when compare two sequences, what we will pay more close attention to is the similarity between sequence shape, and in most cases, the similar shape has different indexes. DTW is a ingenious methods which can overcome this weakness. We can define the distance matrix which can reflect the distance and Alignment relationship, the key process is to find a warp path which wriggles from one corner to the opposite corner, and we can get the DTW distance by a simple iteration. Furthermore, we introduce several methods to accelerate DTW. By some experiment with artificial and official data, we can know, in most cases, DTW method is more accurate than Euclidean distance.

  While DTW considers the whole sequence, there is a method which makes use of local features named Shapelet. By searching the most representative subsequence which is Shapelet, measuring the distance between shapelet and the subsequence of waiting-classify sequence, we can finish classification task.
\end{enabstract}
\enkeywords{sequence classification, distance, DTW method, Shapelets} 

% 往PDF属性里面写下关键词信息
\makeatletter
\XDU@setpdf@keywords
\makeatother

\maketitle

% 文章主要部分
\chapter{绪论}
\section{序列以及序列分类}
现实生活中会有各种各样的序列。在语音识别领域,一段语音可以被看作一段序列;在经济领域,股票的走势可以被看作是一段序列;在分子生物学领域,蛋白质、DNA也可以看成是一段序列;在网络管理中,客户端的登录活动也可以被记录成一段序列。而序列分类在现实生活中也有着广泛的应用。以语音库中的不同样本的语音为训练集,通过对需要测试的语音进行分类,可以识别出语音的来源;在医疗领域,通过对心电图序列的分类,可以得到这个疑似患者的患病信息;在信息检索的研究中,利用序列分类可将文件归类;通过蛋白质、DNA序列的分类,有助于科学家探索蛋白质、DNA的新功能\cite{Xing2010}。

一般说来,序列可以看成是事件的集合,而每一个事件可以由符号或者能比较大小的实数表示。比如一段DNA序列可以表示成$ACCCCCGT$,一段简单的时间序列可以表示成$\left\langle {0.2{\rm{,}}0.3{\rm{,}}0.5{\rm{,}}0.9{\rm{,}} \cdots } \right\rangle $,两种序列的不同之处在于实数的大小是可以度量的,而一般而言符号之间的可度量性比较差,本文主要关注的是实数序列。对于符号序列,有一类的处理方法是借鉴实数序列,比如对于蛋白质序列和DNA序列,有一种基于序列对齐的距离度量方法\cite{kajan2006application}。

对于一般的序列分类任务来说,训练集里面的序列都含有描述其类属性的标签。对于语音识别任务来说,语料库里的语音其标签是其发出者;对于文档归类任务来说,文档的标签即是其所属类别。定义$L$为标签集,$S$为待分类序列集,序列分类的任务就是通过对含有标签信息的训练集$T$的训练,得到一个可将$S$映射到$L$的分类器$C$。可以表示成$C:s \to l,s \in S,l \in L$。对于传统的分类任务来说,常用的分类器主要有$k$NN,决策树、支持向量机\cite{Wu2008}。

\section{序列分类的主要方法}
对于序列分类问题,因为其灵活性,很多的学者从不同的角度提出了大量的方法\cite{Al-Naymat2009}\cite{Keogh2000}\cite{Salvador2007}
\cite{Thrun2000}\cite{Xi2006}\cite{Ye2009}。本文主要介绍其中两种比较典型的方法,一种是基于序列距离度量的Dynamic Time Warping(DTW)方法\cite{Al-Naymat2009}\cite{Batista2011}\cite{Keogh2000}\cite{LESLIE2001}\cite{Lin2007}\cite{Salvador2007}
\cite{Xi2006},另一种是基于特征选择的shapelets方法\cite{Ye2009}。其中对DTW方法做重点介绍。
\subsection{DTW方法}
对于分类任务而言,一种广泛使用而且特别简单的方法是$k$NN方法,使用$k$NN方法时,关键的是测试数据和训练数据的亲近与否,也就是需要一种距离的度量。对于一般的实数序列,最常用的度量是欧氏距离,对于序列$s$和$s^{'}$,它们的欧式距离是:
\[dist\left( {s,{s^{'}}} \right) = \sqrt {\sum\limits_{i = 1}^L {{{\left( {s\left[ i \right] - {s^{'}}\left[ i \right]} \right)}^2}} } \]
同样地,还可以定义其他诸如街区距离,最大值距离。

但是对于序列分类而言,这些距离有很大局限性,主要体现在以下两点:
\begin{enumerate}
  \item 以上的距离度量都要求两序列长度相同,而这一点在实际问题中很难得到满足。比如在基因组分析中,很难保证基因序列的长度相同。
  \item 设想这样一种情况。在步态分析中,同一测试者的步速可能不同,或者在某时间段上存在着加速和减速。那么对于其两段步态序列,比较相似的步态之间可能会有一定的时间差,而上面的这些距离测度只会将同一时刻的步态相比较。也就是说,上面的这些距离测度不能反映出序列比较中的错位。
\end{enumerate}

而DTW方法能够很好地克服上述的两个缺点。

图\ref{fig:1}是DTW示意图,图中展现了长序列中有一定时间错位的两子序列的对齐情况\cite{Giorgino2009}。
\begin{figure}
  \centering
  \includegraphics[width=0.6\textwidth]{./figure/two_way_plot.png}
  \caption{DTW示意图}\label{fig:1}
\end{figure}

图中的虚线表示了两序列点的对齐情况,这里对齐的两点指的是用来比较距离的两点,在传统的距离度量中,对齐的两点是索引值相同的两点。而从图中可以看出,对于DTW而言,点的对齐还参考了点邻近曲线的形态,而不仅仅局限于相同的索引,序列上的某点可以和另一个序列中的多个点对齐。因而DTW方法还能够直接处理两序列长度不同的情况。当然,对于两不同长度的时间序列,可以通过插值的方法使得序列等长,从而应用传统的距离度量方法。比如,对于两序列较短的那个,相邻点的之间可插入一定数目的点,插入点的数值可以用这两点的均值代替,以此小技巧,使得短序列扩展到与长序列长度。关于DTW的对齐规则的得到以及其他知识在后文中会详细说明。

\subsection{Shapelets方法}
Shapelets方法最早是由(Ye $et~al$ 2009)提出,作者经过大量实验,验证了该方法的优秀性\cite{Shapelets_website}。

直观来说,序列的Shapelets是该序列中最能够代表该序列的一段,如图\ref{fig:2}所示。

\begin{figure}
  \centering
  \includegraphics[width=0.6\textwidth]{./figure/shapelets.png}
  \caption{Shapelets示意图}\label{fig:2}
\end{figure}

图中的样本是两个徽章,徽章的边缘可以转换成序列。对于这两个徽章而言,徽章的下部的轮廓最能够反映该徽章的特征,因而徽章下部轮廓对应的序列是徽章序列的Shapelets。可以看出,两徽章的下部明显不同,因而可以由下部序列之间的差异,利用分类器,对徽章进行分类。在这里,一般最常用的分类器是$k$NN分类器,它简单而有效,通常取$k=1$\cite{Xi2006}

Shapelets方法有着广泛的应用,比如在植物学中,利用叶柄与叶片的角度,对植物进行分类;在考古学领域,通过对出土器件的特征部位的分类,能得到考古学发现\cite{anthropology}。


% 致谢
\begin{thanksfor}
    这次的毕业论文设计总结是在我的指导老师徐林莉老师亲切关怀和悉心指导下完成的。从毕业设计选题到设计完成,徐老师给予了我耐心指导与细心关怀,在此表示感谢。徐老师有严肃的科学态度,严谨的治学精神和精益求精的工作作风,这些都是我所需要学习的!感谢研究组里提供的优良的科研资源,这让我享受到了难忘的学习时光,感谢师兄们的指导。另外,也特别感谢崔元顺同学提供的latex模板。
\end{thanksfor} 

%参考文献
\phantomsection%生成该页的链接
\addcontentsline{toc}{chapter}{\bibname}
\bibliographystyle{XDUbib}%plain ieeetr
\bibliography{ThesisFiles/RefFile}%在正文中必须引用,才能显示对应的参考文献

% 附录部分
\appendix
\input{ThesisFiles/Appendix}

\end{document}

